% vim: autoindent sw=2 sts=2 ts=2 fdm=marker spell spelllang=en_gb iskeyword+=_ efm=%f\:%l\:\ %m,nuweb\:\ %m\ (%f\\,\ %l)
\documentclass[12pt,a4paper]{article}
\usepackage[T1]{fontenc}
\usepackage{fullpage}
\usepackage{textcomp}
\usepackage{hyperref}
\usepackage{upquote}

\title{\vspace{-48pt}
  UROP 2014 Tickl Project
}
\author{Isaac Dunn
  \and Robert Kovacsics
  \and Tom Lefley
  \and Katie Scott
  \and Raahil Shah
  \and Alexander Simpson
}

\newcommand{\tomcatwd}{{\tt /var/\hspace{0pt}lib/\hspace{0pt}tomcat7}}
\newcommand{\localHardcodedJs}{{\tt urop-2014-ticking-ui/\hspace{0pt}ETick/\hspace{0pt}WebContent/\hspace{0pt}js/\hspace{0pt}hardcoded.js}}
\newcommand{\deployedHardcodedJs}{\tomcatwd{\tt/\hspace{0pt}webapps/\hspace{0pt}\{deployed name\}/\hspace{0pt}js/\hspace{0pt}hardcoded.js}}
\newcommand{\gitoliteRc}{{\tt \~{}git/\hspace{0pt}.gitolite.rc}}
\newcommand{\gitoliteAuthkeys}{{\tt \~{}git/\hspace{0pt}.ssh/\hspace{0pt}authorized\_keys}}

\usepackage{fullpage}

\begin{document}
\maketitle
\section{Installing and configuring}
The steps in this manual were tested using a new server image from \url{http://www.ubuntu.com/}, version 14.04.1 (Trusty) using QEMU 2.1.0 and also a virtual machine on the DTG server.
The necessary programs for the running of the server are:
\begin{itemize}
  \item Apache Tomcat version 7 ({\tt tomcat7}); server used in the project.
  \item Maven version 3 ({\tt maven}); tool to compile the project.
  \item Java Development Kit version 7 ({\tt default-jdk}); compiler for the project.
  \item NodeJS package manager ({\tt npm}); to install bower which fetches JavaScript dependencies.
  \item NodeJS runtime ({\tt nodejs}); needed to run bower.
  \item Git version control ({\tt git}); acts as a database for user's submissions.
  \item Gitolite version 3 ({\tt gitolite3}); adds on user based control for git.
  \item Secure Shell ({\tt ssh}); used for authentication in gitolite.
  \item MongoDB database ({\tt mongodb}); database used in the project.
  \item Restricted shell ({\tt rssh}); for the {\tt tomcat7} user.
  \item A server using Andrew Rice's dynamic tester; for the dynamic testing of ticks.
\end{itemize}

We also need to configure the system a bit more, before the project can be deployed, the necessary steps are:
\begin{itemize}
  \item Make {\tt rssh} the shell of {\tt tomcat7} with {\tt sudo chsh tomcat7 -{}-shell /usr/bin/rssh}.

  \item If the gitolite installation did not already do so, create a user for gitolite, with {\tt sudo useradd -{}-system -{}-user-group -{}-create-home -{}-base-dir /var/lib -{}-shell /bin/bash git} (you may wish to have the {\tt base-dir} on {\tt /local/data}, if it is a different mountpoint).
    Also, under the user {\tt git}, run {\tt gitolite setup -pk Your public SSH key}.

  \item Then clone the gitolite administration repository with {\tt git clone git@localhost:\hspace{0pt}gitolite-admin.git} and edit {\tt conf/gitolite.conf} to have {\tt include \textquotedbl{}/\hspace{0pt}var/\hspace{0pt}lib/\hspace{0pt}tomcat7/\hspace{0pt}webapps/\hspace{0pt}urop\_gitolite.conf\textquotedbl} and push these changes.

  \item You may also wish to put MongoDB onto a different path, which can be done in {\tt /etc/mongodb.conf}

  \item Add {\tt tomcat7} to group {\tt git}, so that it can add SSH keys, read repositories with {\tt usermod -{}-append -{}-groups git tomcat7}.

  \item To allow Tomcat to use gitolite, change the permissions on {\tt \~{}git/.gitolite/*} to {\tt 0770}.

  \item Similarly, we need to make gitolite use a {\tt umask} of {\tt 0007}, so that created files (repositories) can be removed by Tomcat.
    This can be done in \gitoliteRc{}.

  \item We want to be able to send a command to read the gitolite keys into \gitoliteAuthkeys{}, so add {\tt SSH\_AUTHKEYS => [ \textquotesingle{}post-compile/ssh-authkeys\textquotesingle{} ],} to the {\tt \%RC} section of \gitoliteRc{}.

  \item However, to avoid the need to change the {\tt umask} of Apache, the Tickl system tries to avoid creating files for gitolite (exception is adding SSH keys, as gitolite won't need to delete them).
    So we use an SSH command to create repositories, so include {\tt command=\textquotedbl{}\~{}git{}/configure\_gitolite.sh tomcat7\textquotedbl{} Tomcat7 SSH public key} in \gitoliteAuthkeys{}.
    Do this outside of the {\tt gitolite start/end} comment blocks.
    You should also connect from {\tt tomcat7} to {\tt git} using SSH (with the {\tt RSA} key), so that the host key is known, by running {\tt ssh git@localhost -o HostKeyAlgorithm=ssh-rsa}.

  \item The contents of {\tt \~{}git{}/configure\_gitolite.sh} should be
    \begin{verbatim}
#!/bin/sh
test -n "$SSH_ORIGINAL_COMMAND" || exit 1;

CMD=`grep "$SSH_ORIGINAL_COMMAND" <<'EOF' | head -n1
compile
trigger POST_COMPILE
trigger SSH_AUTHKEYS
list-dangling-repos
EOF
`

if [ -z "$CMD" ]; then
  # Trying it as a gitolite action
  /usr/share/gitolite3/gitolite-shell "$1"
else
  gitolite $CMD
fi
    \end{verbatim}
    Make sure it is an executable file.

  \item We use slashes in path parameters, so make sure Tomcat allows us to do so, include {\tt org.apache.tomcat.util.buf.UDecoder.ALLOW\_ENCODED\_SLASH=true} in {\tt /etc/tomcat7/catalina.properties}.
    
  \item You may also wish to set {\tt MaxSessions} in {\tt /etc/ssh/sshd\_config} to something higher than 10, higher level of concurrency with SSH\@.

  \item Finally, we need to trust the certificate with which the dynamic tester is configured.
    Have the following lines in {\tt /etc/default/tomcat7} (replace {\tt TRUSTSTORE}, {\tt CERTIFICATE} and {\tt PASSWORD}).
    \begin{verbatim}
# Trust Andy's dynamic tester's SSL certificate too.
# Replace PASSWORD with one of your own.
JAVA_OPTS="$JAVA_OPTS -Djavax.net.ssl.trustStore=TRUSTSTORE"
JAVA_OPTS="$JAVA_OPTS -Djavax.net.ssl.trustStorePassword=PASSWORD"
    \end{verbatim}

    The file {\tt TRUSTSTORE} can be generated with {\tt keytool -import -trustcacerts -alias containers -file CERTIFICATE -keystore {\tt TRUSTSTORE}}, giving it the {\tt PASSWORD} as above.
\end{itemize}

Then the source files for the Tickl software can now be downloaded with the following three commands.
\begin{verbatim}
git clone --depth 1 https://github.com/ucam-cl-dtg/urop-2014-git.git
git clone --depth 1 https://github.com/ucam-cl-dtg/urop-2014-signup.git
git clone --depth 1 https://github.com/ucam-cl-dtg/urop-2014-tester.git
git clone --depth 1 https://github.com/ucam-cl-dtg/urop-2014-ticking-ui.git
\end{verbatim}

To download the dependencies of these projects, you need to be able to access the {\tt maven.dtg.cl.cam.ac.uk} server and have the following in your {\tt \~{}/.m2/settings.xml}.
\begin{verbatim}
<settings>
  <profiles>
    <profile>
      <id>repositories</id>
      <repositories>
        <repository>
          <id>dtg-repository</id>
          <url>sftp://maven@maven.dtg.cl.cam.ac.uk/mirror/</url>
          <releases>
            <enabled>true</enabled>
          </releases>
          <snapshots>
            <enabled>true</enabled>
          </snapshots>
        </repository>
      </repositories>
    </profile>
  </profiles>
  <activeProfiles>
    <activeProfile>repositories</activeProfile>
  </activeProfiles>
</settings>
\end{verbatim}

First, before creating a war package, install bower (JavaScript dependency downloader) using {\tt npm install -g bower}, change into {\tt urop-2014-ticking-ui/ETick} and run {\tt bower install}.
If you get {\tt /usr/bin/env: node: No such file or directory}, you need to do {\tt sudo ln -s /usr/bin/nodejs /usr/bin/node}.
This downloads all the JavaScript dependencies that we need.
Also, because Tomcat serves the front-end webpages too, the hard-coded strings can not be in an external configuration file, they are in \localHardcodedJs, you may want to set it now (you can also just edit \deployedHardcodedJs, but a re-deploy will overwrite those changes).

Then for each of the directories {\tt urop-2014-git/server}, {\tt urop-2014-signup/core}, {\tt urop-2014-tester/private} and {\tt urop-2014-ticking-ui/ETick} you want to change into it and run {\tt mvn clean package}.
Installing each war file is then just a matter of moving it into \tomcatwd{}{\tt/webapps}.

The sample configuration files for the project are in {\tt urop-2014-ticking-ui/defaults}, copy them to \tomcatwd{}{} and edit them to suit the set-up.

\section{Troubleshooting}
\begin{description}
  \item[FATAL\@: errors found but the logfile could not be created] happen when the user {\tt git} has not got correct permissions or ownership on its home directory (or where it stores the logs).

  \item[NoClassDefFoundError{\rm s and} ClassNotFoundException{\rm s}] may arise due to loading old class files, if a different version was compiled. Run {\tt mvn clean} and also delete {\tt WebContent/WEB-INF/lib} and {\tt WebContent/WEB-INF/classes} before re-installing.

  \item[Unable to initialise extensions] happened to me because I accidentally installed {\tt maven2}, not {\tt maven}.

\end{description}
\end{document}
