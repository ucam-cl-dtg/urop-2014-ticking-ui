% vim: autoindent sw=2 sts=2 ts=2 fdm=marker spell spelllang=en_gb iskeyword+=_ efm=%f\:%l\:\ %m,nuweb\:\ %m\ (%f\\,\ %l)
\documentclass[12pt,a4paper]{article}
\usepackage[T1]{fontenc}
\usepackage{fullpage}
\usepackage{hyperref}

\title{\vspace{-48pt}
  UROP 2014 Tickl Project
}
\author{Isaac Dunn
  \and Robert Kovacsics
  \and Tom Lefley
  \and Katie Scott
  \and Raahil Shah
  \and Alexander Simpson
}

\newcommand{\tomcatwd}{{\tt /var/\hspace{0pt}lib/\hspace{0pt}tomcat7}}
\newcommand{\localHardcodedJs}{{\tt urop-2014-ticking-ui/\hspace{0pt}ETick/\hspace{0pt}WebContent/\hspace{0pt}js/\hspace{0pt}hardcoded.js}}
\newcommand{\deployedHardcodedJs}{\tomcatwd{\tt/\hspace{0pt}webapps/\hspace{0pt}\{deployed name\}/\hspace{0pt}js/\hspace{0pt}hardcoded.js}}

\usepackage{fullpage}

\begin{document}
\maketitle
\section{Installing and configuring}
The steps in this manual were tested using a new server image from \url{http://www.ubuntu.com/}, version 14.04.1 (Trusty) using QEMU 2.1.0 and also a virtual machine on the DTG server.
The necessary programs for the running of the server are:
\begin{itemize}
  \item Apache Tomcat version 7 ({\tt tomcat7}); server used in the project.
  \item Maven version 3 ({\tt maven}); tool to compile the project.
  \item Java Development Kit version 7 ({\tt default-jdk}); compiler for the project.
  \item NodeJS package manager ({\tt npm}); to install bower which fetches JavaScript dependencies.
  \item NodeJS runtime ({\tt nodejs}); needed to run bower.
  \item Git version control ({\tt git}); acts as a database for user's submissions.
  \item Gitolite version 3 ({\tt gitolite3}); adds on user based control for git.
  \item MongoDB database ({\tt mongodb}); database used in the project.
  \item Restricted shell ({\tt rssh}); for the {\tt tomcat7} user.
  \item A server using Andrew Rice's dynamic tester; for the dynamic testing of ticks.
\end{itemize}

Then the source files for the Tickl software can now be downloaded with the following three commands.
\begin{verbatim}
git clone --depth 1 https://github.com/ucam-cl-dtg/urop-2014-git.git
git clone --depth 1 https://github.com/ucam-cl-dtg/urop-2014-signup.git
git clone --depth 1 https://github.com/ucam-cl-dtg/urop-2014-tester.git
git clone --depth 1 https://github.com/ucam-cl-dtg/urop-2014-ticking-ui.git
\end{verbatim}

To download the dependencies of these projects, you need to be able to access the {\tt maven.dtg.cl.cam.ac.uk} server and have the following in your {\tt \~{}/.m2/settings.xml}.
\begin{verbatim}
<settings>
  <profiles>
    <profile>
      <id>repositories</id>
      <repositories>
        <repository>
          <id>dtg-repository</id>
          <url>sftp://maven@maven.dtg.cl.cam.ac.uk/mirror/</url>
          <releases>
            <enabled>true</enabled>
          </releases>
          <snapshots>
            <enabled>true</enabled>
          </snapshots>
        </repository>
      </repositories>
    </profile>
  </profiles>
  <activeProfiles>
    <activeProfile>repositories</activeProfile>
  </activeProfiles>
</settings>
\end{verbatim}

First, before creating a war package, install bower (JavaScript dependency downloader) using {\tt npm install -g bower}, change into {\tt urop-2014-ticking-ui/ETick} and run {\tt bower install}.
If you get {\tt /usr/bin/env: node: No such file or directory}, you need to do {\tt sudo ln -s /usr/bin/nodejs /usr/bin/node}.
This downloads all the JavaScript dependencies that we need.
Also, because Tomcat serves the front-end webpages too, the hard-coded strings can not be in an external configuration file, they are in \localHardcodedJs, you may want to set it now (you can also just edit \deployedHardcodedJs, but a re-deploy will overwrite those changes).

Then for each of the directories {\tt urop-2014-git/server}, {\tt urop-2014-signup/core}, {\tt urop-2014-tester/private} and {\tt urop-2014-ticking-ui/ETick} you want to change into it and run {\tt mvn clean package}.
Installing each war file is then just a matter of moving it into \tomcatwd{}{\tt/webapps}.

The sample configuration files for the project are in {\tt urop-2014-ticking-ui/defaults}, copy them to \tomcatwd{}{} and edit them to suit the set-up.

\section{Troubleshooting}
\begin{description}
  \item[NoClassDefFoundError{\rm s and} ClassNotFoundException{\rm s}] may arise due to loading old class files, if a different version was compiled. Run {\tt mvn clean} and also delete {\tt WebContent/WEB-INF/lib} and {\tt WebContent/WEB-INF/classes} before re-installing.

  \item[Unable to initialise extensions] happened to me because I accidentally installed {\tt maven2}, not {\tt maven}.

\end{description}
\end{document}
